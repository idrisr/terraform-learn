\documentclass[openany, 12pt]{book}
\makeindex

\usepackage{amsmath}
\usepackage{amssymb}
\usepackage{booktabs}
\usepackage{csvsimple-l3}
\usepackage{bussproofs}
\usepackage{dirtytalk}
\usepackage[dvipsnames]{xcolor}
\usepackage{enumitem}
\usepackage{epigraph}
\usepackage{forest}
\usepackage{formal-grammar}
\usepackage{graphicx}
\usepackage[citecolor=blue,colorlinks=true, linkcolor=blue, urlcolor=blue]{hyperref}
\usepackage{kantlipsum}
\usepackage{makeidx}
\usepackage[margin=0.8in]{geometry}
\usepackage{mathrsfs}
\usepackage{minted}
\usepackage{multicol}
\usepackage{standalone}
\usepackage[style=authortitle]{biblatex}
\usepackage[T1]{fontenc}
\usepackage[tableaux]{prooftrees}
\usepackage{tcolorbox}
\usepackage{tikz}
\usepackage{titlesec}
\usepackage{xcolor}

\usetikzlibrary{arrows}
\usetikzlibrary{arrows.meta}
\usetikzlibrary{automata}
\usetikzlibrary{calc}
\usetikzlibrary{fit}
\usetikzlibrary{petri}
\usetikzlibrary{positioning}

\tcbuselibrary{breakable}
\tcbuselibrary{listings}
\tcbuselibrary{minted}
\tcbuselibrary{skins}
\tcbuselibrary{theorems}

\newcounter{filePrg}

% \addbibresource{biblio.bib}
\setlength{\parindent}{0pt}

\renewcommand{\emph}[1]{\textit{#1}}
\setlength{\parindent}{0pt}

\newcommand\setboxcounter[2]{\setcounter{tcb@cnt@#1}{#2}}
\setlength{\parindent}{10pt}
\newcommand{\set}[1]{\{#1\}}

\definecolor{CaribbeanBlue}{RGB}{0, 206, 209} % Define Caribbean Blue
\NewTcbTheorem[list inside=definition]{definition}
{Definition}{
	breakable,
	colback=CaribbeanBlue!05,
	colframe=CaribbeanBlue!35!black,
	fonttitle=\bfseries}{th}

\NewTcbTheorem[list inside=intuition]{intuition}{Intuition}{
	breakable,
	colback=blue!5,
	colframe=blue!35!black,
	fonttitle=\bfseries}{th}

\NewTcbTheorem{example}{Example}{
	breakable,
	colback=white,
	colframe=green!35!black,
	fonttitle=\bfseries}{th}

\NewTcbTheorem{verify}{Verify}{
	breakable,
	float,
	colback=red!5,
	colframe=red!35!black,
	fonttitle=\bfseries}{th}

\NewTcbTheorem[list inside=theorem]{theorem}{Theorem}{
	breakable,
	colback=gray!10,
	colframe=gray!35!black,
	fonttitle=\bfseries}{th}

\NewTcbTheorem[
	list inside=exercise,
	number within=section
]
{exercise}{Exercise}{
	breakable,
	colback=white,
	colframe=black,
	fonttitle=\bfseries}{th}

\newcommand{\hask}[1]{\mintinline{haskell}{#1}}

\newenvironment{alist}
{\begin{enumerate}[label={*}, leftmargin=*, itemsep=0pt, parsep=0pt]}
		{\end{enumerate}}

\newenvironment{blist}
{\begin{enumerate}[label={}, leftmargin=*, itemsep=0pt, parsep=0pt]}
		{\end{enumerate}}


\renewcommand{\thesection}{\arabic{section}}
\tcbset{enhanced jigsaw}

\newtcbinputlisting{\codeFromFile}[2]{
	listing file={#1},
	listing engine=minted,
	minted style=colorful,
	minted language=haskell,
	minted options={breaklines,linenos,numbersep=3mm},
	colback=blue!5!white,colframe=blue!75!black,listing only,
	left=5mm,enhanced,
	title={#2},
	overlay={\begin{tcbclipinterior}\fill[red!20!blue!20!white] (frame.south west)
				rectangle ([xshift=5mm]frame.north west);\end{tcbclipinterior}}
}

\newtcblisting{haskell}[1]
{
	listing engine=minted,
	minted style=colorful,
	minted language=haskell,
	minted options={breaklines,linenos,numbersep=3mm},
	colback=blue!5!white,colframe=blue!75!black,listing only,
	left=5mm,enhanced,
	title={#1},
	overlay={\begin{tcbclipinterior}\fill[red!20!blue!20!white] (frame.south west)
				rectangle ([xshift=5mm]frame.north west);\end{tcbclipinterior}}
}

\newtcblisting{shell}[1]
{
	listing engine=minted,
	minted style=colorful,
	minted language=shell,
	minted options={breaklines,linenos,numbersep=3mm},
	colback=blue!5!white,colframe=blue!75!black,listing only,
	left=5mm,enhanced,
	title={#1},
	overlay={\begin{tcbclipinterior}\fill[red!20!blue!20!white] (frame.south west)
				rectangle ([xshift=5mm]frame.north west);\end{tcbclipinterior}}
}

\title{Terraform Learn via AI}
\author{Idris}
\date{May 2025}

% chktex-file 36
\begin{document}

\chapter{Terraform Language}
The HCL language is the terraform language.

\begin{definition}{blocks}{}
	Everything in terraform is blocks. The language is block structured.

	\begin{alist}
		\item blocks are containers for other content
		\item usually represent the configuration of some kind of object, like a
		resource.
		\item have a block type
		\item have zero or more labels,
		\item have a body that contains any number of arguments and nested blocks
	\end{alist}
	Most of terraform's features are controlled by top-level blocks in a
	configuration file.

\end{definition}

\begin{definition}{labels}{}
	The semantics of label is based of its index.
\end{definition}

\begin{definition}{argument}{}
	Arguments assign a value to a name. They appear within blocks.

	An argument consists of a left-hand side (LHS), which is a name, and a
	right-hand side (RHS), which is an expression. The expression is assigned to
	the name.

	Some blocks, such as variable and output, may include an argument named
	type, which declares a type constraint for the value. In other blocks, such
	as resource, the concept of type is instead conveyed via the first label
	following the block type.
\end{definition}

\begin{definition}{expression}{}
	represent a value, either literally or by referencing and combining other
	values. They appear as values for arguments, or within other expressions.
\end{definition}

\begin{definition}{function}{}
	\begin{alist}
		\item built-in operation that takes zero or more input expressions and
		returns an output value.
		\item pure---produce same result for the same inputs and have no side effects
		\item two exceptions: uuid() and timestamp()
		\item cannot access external state or cause external effects.
	\end{alist}
\end{definition}

\begin{shell}{}
<block-type> "<label-1>" "<label-2>" {
    argument = expression
    nested_block {
        ... 
    }
}
\end{shell}

\begin{table}
	\centering
	\begin{tabular}{lll}
		\toprule
		{Block Type} & {Label Index} & {Meaning}                                              \\
		\midrule
		resource     & 0             & Resource type                                          \\
		             & 1             & local name---identifier within module                  \\
		provider     & 0             & provider name (aws, google)                            \\
		module       & 0             & module name                                            \\
		variable     & 0             & variable name (input identifier)                       \\
		output       & 0             & output name (used in output map)                       \\
		data         & 0             & data source type (aws\_ami)                            \\
		             & 1             & local name                                             \\
		for\_each    &               & not a block---argument used to iterate over collection \\
		\bottomrule
	\end{tabular}
	\caption{Label semantics by block type in Terraform}
\end{table}

\begin{table}[h!]
	\centering
	\begin{tabular}{ll}
		\toprule
		category           & example functions              \\
		\midrule
		numeric            & abs, ceil, floor, max, min     \\
		string             & upper, lower, replace, trim    \\
		collection         & length, join, merge, flatten   \\
		encoding           & base64encode, jsonencode       \\
		filesystem         & dirname, basename, file        \\
		date/time          & timestamp, timeadd, formatdate \\
		hash/crypto        & sha256, md5, uuidv5            \\
		ip network         &                                \\
		type conversion    & type, can, try, contains       \\
		terraform specific &                                \\
		\bottomrule
	\end{tabular}
	\caption{Common Terraform built-in function categories}
\end{table}

\begin{definition}{provider}{}

\end{definition}

\chapter{Terraform CLI}

% \printbibliography{}
% \printindex{}
\end{document}
