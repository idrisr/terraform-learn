\documentclass[openany, 12pt]{book}
\makeindex

\usepackage{amsmath}
\usepackage{amssymb}
\usepackage{booktabs}
\usepackage{csvsimple-l3}
\usepackage{bussproofs}
\usepackage{dirtytalk}
\usepackage[dvipsnames]{xcolor}
\usepackage{enumitem}
\usepackage{epigraph}
\usepackage{forest}
\usepackage{formal-grammar}
\usepackage{graphicx}
\usepackage[citecolor=blue,colorlinks=true, linkcolor=blue, urlcolor=blue]{hyperref}
\usepackage{kantlipsum}
\usepackage{makeidx}
\usepackage[margin=0.8in]{geometry}
\usepackage{mathrsfs}
\usepackage[outputdir=build]{minted}
\usepackage{multicol}
\usepackage{standalone}
\usepackage[style=authortitle]{biblatex}
\usepackage[T1]{fontenc}
\usepackage[tableaux]{prooftrees}
\usepackage{tcolorbox}
\usepackage{tikz}
\usepackage{titlesec}
\usepackage{xcolor}

\usetikzlibrary{arrows}
\usetikzlibrary{arrows.meta}
\usetikzlibrary{automata}
\usetikzlibrary{calc}
\usetikzlibrary{fit}
\usetikzlibrary{petri}
\usetikzlibrary{positioning}

\tcbuselibrary{breakable}
\tcbuselibrary{listings}
\tcbuselibrary{minted}
\tcbuselibrary{skins}
\tcbuselibrary{theorems}

\newcounter{filePrg}

% \addbibresource{biblio.bib}
\setlength{\parindent}{0pt}

\renewcommand{\emph}[1]{\textit{#1}}
\setlength{\parindent}{0pt}

\newcommand\setboxcounter[2]{\setcounter{tcb@cnt@#1}{#2}}
\setlength{\parindent}{10pt}
\newcommand{\set}[1]{\{#1\}}

\definecolor{CaribbeanBlue}{RGB}{0, 206, 209} % Define Caribbean Blue
\NewTcbTheorem[list inside=definition]{definition}
{Definition}{
	breakable,
	colback=CaribbeanBlue!05,
	colframe=CaribbeanBlue!35!black,
	fonttitle=\bfseries}{th}

\NewTcbTheorem[list inside=intuition]{intuition}{Intuition}{
	breakable,
	colback=blue!5,
	colframe=blue!35!black,
	fonttitle=\bfseries}{th}

\NewTcbTheorem{example}{Example}{
	breakable,
	colback=white,
	colframe=green!35!black,
	fonttitle=\bfseries}{th}

\NewTcbTheorem{verify}{Verify}{
	breakable,
	float,
	colback=red!5,
	colframe=red!35!black,
	fonttitle=\bfseries}{th}

\NewTcbTheorem[list inside=theorem]{theorem}{Theorem}{
	breakable,
	colback=gray!10,
	colframe=gray!35!black,
	fonttitle=\bfseries}{th}

\NewTcbTheorem[
	list inside=exercise,
	number within=section
]
{exercise}{Exercise}{
	breakable,
	colback=white,
	colframe=black,
	fonttitle=\bfseries}{th}

\newcommand{\hask}[1]{\mintinline{haskell}{#1}}

\newenvironment{alist}
{\begin{enumerate}[label={*}, leftmargin=*, itemsep=0pt, parsep=0pt]}
		{\end{enumerate}}

\newenvironment{blist}
{\begin{enumerate}[label={}, leftmargin=*, itemsep=0pt, parsep=0pt]}
		{\end{enumerate}}


\renewcommand{\thesection}{\arabic{section}}
\tcbset{enhanced jigsaw}

\newtcbinputlisting{\codeFromFile}[2]{
	listing file={#1},
	listing engine=minted,
	minted style=colorful,
	minted language=haskell,
	minted options={breaklines,linenos,numbersep=3mm},
	colback=blue!5!white,colframe=blue!75!black,listing only,
	left=5mm,enhanced,
	title={#2},
	overlay={\begin{tcbclipinterior}\fill[red!20!blue!20!white] (frame.south west)
				rectangle ([xshift=5mm]frame.north west);\end{tcbclipinterior}}
}

\newtcblisting{haskell}[1]
{
	listing engine=minted,
	minted style=colorful,
	minted language=haskell,
	minted options={breaklines,linenos,numbersep=3mm},
	colback=blue!5!white,colframe=blue!75!black,listing only,
	left=5mm,enhanced,
	title={#1},
	overlay={\begin{tcbclipinterior}\fill[red!20!blue!20!white] (frame.south west)
				rectangle ([xshift=5mm]frame.north west);\end{tcbclipinterior}}
}

\newtcblisting{shell}[1]
{
	listing engine=minted,
	minted style=colorful,
	minted language=shell,
	minted options={breaklines,linenos,numbersep=3mm},
	colback=blue!5!white,colframe=blue!75!black,listing only,
	left=5mm,enhanced,
	title={#1},
	overlay={\begin{tcbclipinterior}\fill[red!20!blue!20!white] (frame.south west)
				rectangle ([xshift=5mm]frame.north west);\end{tcbclipinterior}}
}

\title{Terraform Learn via AI}
\author{Idris}
\date{May 2025}

% chktex-file 36
\begin{document}
\tableofcontents

\part{Theory}
\chapter{Resource Lifecycle}
A resource is the atomic unit in terraform. Resources are the most important
element in the Terraform language. Each resource block describes one or more
infrastructure objects, such as virtual networks, compute instances, or
higher-level components such as DNS records.

\begin{definition}{Managed Resource}{}
	Often cloud-based resources, but they don’t have to be. Anything that can be
	represented as CRUD can be managed as a Terraform resource.
\end{definition}

\begin{example}{Local Only Resources}{}
	\begin{enumerate}[label = {(\arabic*)}]
		\item resources for creating private keys
		\item self-signed TLS certificates
		\item random ids
	\end{enumerate}
\end{example}

There are three different representations of infra. Before each terraform call,
it calls to the api to capture the current state.

\begin{center}
	\begin{tabular}{ll}
		\toprule
		        &                       \\
		\midrule
		config  & tf files              \\
		state   & tfstate               \\
		reality & queried via api calls \\
		\bottomrule
	\end{tabular}
\end{center}

\begin{definition}{attributes}{}
	Every resource has attributes.  Each attribute is either required, optional,
	or read-only.  Required attributes must be present in a resource config
	block, otherwise it's invalid.  Optational attributes can be absent or
	present. Read-only attributes are those set by the provider, and therefore
	only exist when the config is built and deployed.

	\begin{center}
		\begin{tabular}{ll}
			\toprule
			attribute type & official name                  \\
			\midrule
			required       & required argument              \\
			optional       & optional argument              \\
			read-only      & attribute reference (computed) \\
			\bottomrule
		\end{tabular}
	\end{center}
\end{definition}

\chapter{State}
Terraform keeps a state file mapping resources in config to real-world objects
(API IDs, attributes). Drift between state and reality is the core problem it
manages.

\begin{table}[H]
	\centering
	\begin{tabular}{ll}
		\toprule
		\textbf{State} & \textbf{Meaning}                                                   \\
		\midrule
		Planned        & Diff detected between config and state (shown in \texttt{plan})    \\
		Creating       & Provider \texttt{Create} called during \texttt{apply}              \\
		Updating       & Provider \texttt{Update} called to reconcile changes               \\
		Destroying     & Provider \texttt{Delete} called during destroy or replacement      \\
		Stable         & Config matches state; no changes needed                            \\
		Tainted        & Resource marked for recreation in next \texttt{apply}              \\
		Unknown        & Attributes missing or refresh failed (shown as \texttt{<unknown>}) \\
		\bottomrule
	\end{tabular}
	\caption{Terraform resource lifecycle states}
\end{table}


\begin{definition}{State Management}{}
	Terraform is a state management tool that performs CRUD operations (create,
	read, update, delete) on managed resources.
\end{definition}

\chapter{Dependency graph}
Resources form a DAG, so Terraform knows what to create/destroy in order and in
parallel.

\chapter{Modules}
Modules are reusable abstractions over resources; how you scale beyond one-off configs.
A module can create many resources. In this way, a module is function whose
codomain is a list of resources.

\begin{definition}{module}{}
	A \emph{module} is a reusable abstraction over resources. It accepts
	\emph{inputs} (variables) and yields a finite set of \emph{resources}
	(and possibly other modules).

	Formally, a module behaves like a function
	\begin{align*}
		M : \text{Inputs} \to \{\text{resources}\}
	\end{align*}

	where the codomain is a collection of resources, which are then merged
	into the global resource dependency graph.
\end{definition}


\chapter{Providers}
Each resource type lives in a provider. A provider is a plugin for Terraform
that offers a collection of resource types. Each resource type is implemented by
a provider. A provider provides resources to manage a single cloud or
on-premises infrastructure platform. Providers are distributed separately from
Terraform, but Terraform can automatically install most providers when
initializing a working directory.

\begin{definition}{terraform provider}{}
	\begin{alist}
		\item plugins for the terraform core
		\item each plugin is written in go to implement a specific interface
		\item core knows how to install and execute the plugin
		\item each designed to work with some platform---aws, azure, google cloud
		\item core communicates with plugins via remote procedure calls---rpcs
		\item plugins communicate with their corresponding platforms via the network
	\end{alist}

\end{definition}

\part{Practice}
\chapter{Tools}
\begin{definition}{terraform core}{}
	\begin{alist}
		\item terraform binary
		\item provides functionality used by all platforms
		\item cli, parser and interpreter for Terraform code (HCL)
		\item ability to build a dependency graph from resources and data sources
		\item logic to read and write state files, and so on
	\end{alist}
\end{definition}

\begin{intuition}{Terraform}{}
	There are two programs---terraform and then a provider. Terraform interprets
	your config file, then sends RPC messages to the provider. The provider then
	makes the network calls via API.\@
	\begin{haskell}{}
core     :: TF -> RPC
provider :: RPC -> HTTP
	\end{haskell}
\end{intuition}

\begin{definition}{terraform plan}{}
	\begin{enumerate}[label = {(\arabic*)}]
		\item read the configuration and state
		\item determine actions to take. can be one of
		      \begin{multicols}{5}
			      \begin{enumerate}[label = {(\arabic*)}]
				      \item create
				      \item read
				      \item update
				      \item delete
				      \item no-op
			      \end{enumerate}
		      \end{multicols}
		\item output the plan
	\end{enumerate}
\end{definition}

\chapter{Terraform CLI}

\begin{definition}{commands}{}
	\begin{enumerate}[label = {(\arabic*)}]
		\item init
		\item apply
		\item destroy
	\end{enumerate}
\end{definition}

\chapter{Terraform Language}
The HCL language is the terraform language.

\begin{definition}{blocks}{}
	Everything in terraform is blocks. The language is block structured.

	\begin{alist}
		\item blocks are containers for other content
		\item usually represent the configuration of some kind of object, like a
		resource.
		\item have a block type
		\item have zero or more labels,
		\item have a body that contains any number of arguments and nested blocks
	\end{alist}
	Most of terraform's features are controlled by top-level blocks in a
	configuration file.

\end{definition}

\begin{definition}{labels}{}
	The semantics of label is based of its index.
\end{definition}

\begin{definition}{argument}{}
	Arguments assign a value to a name. They appear within blocks.

	An argument consists of a left-hand side (LHS), which is a name, and a
	right-hand side (RHS), which is an expression. The expression is assigned to
	the name.

	Some blocks, such as variable and output, may include an argument named
	type, which declares a type constraint for the value. In other blocks, such
	as resource, the concept of type is instead conveyed via the first label
	following the block type.
\end{definition}

\begin{definition}{expression}{}
	represent a value, either literally or by referencing and combining other
	values. They appear as values for arguments, or within other expressions.
\end{definition}

\begin{definition}{function}{}
	\begin{alist}
		\item built-in operation that takes zero or more input expressions and
		returns an output value.
		\item pure---produce same result for the same inputs and have no side effects
		\item two exceptions: uuid() and timestamp()
		\item cannot access external state or cause external effects.
	\end{alist}
\end{definition}

\begin{shell}{}
<block-type> "<label-1>" "<label-2>" {
	argument = expression
	nested_block {
	...
	}
}
\end{shell}

\begin{table}
	\centering
	\begin{tabular}{lll}
		\toprule
		{Block Type} & {Label Index} & {Meaning}                                              \\
		\midrule
		resource     & 0             & Resource type                                          \\
		             & 1             & local name---identifier within module                  \\
		provider     & 0             & provider name (aws, google)                            \\
		module       & 0             & module name                                            \\
		variable     & 0             & variable name (input identifier)                       \\
		output       & 0             & output name (used in output map)                       \\
		data         & 0             & data source type (aws\_ami)                            \\
		             & 1             & local name                                             \\
		for\_each    &               & not a block---argument used to iterate over collection \\
		\bottomrule
	\end{tabular}
	\caption{Label semantics by block type in Terraform}
\end{table}

\begin{table}[h!]
	\centering
	\begin{tabular}{ll}
		\toprule
		category           & example functions              \\
		\midrule
		numeric            & abs, ceil, floor, max, min     \\
		string             & upper, lower, replace, trim    \\
		collection         & length, join, merge, flatten   \\
		encoding           & base64encode, jsonencode       \\
		filesystem         & dirname, basename, file        \\
		date/time          & timestamp, timeadd, formatdate \\
		hash/crypto        & sha256, md5, uuidv5            \\
		ip network         &                                \\
		type conversion    & type, can, try, contains       \\
		terraform specific &                                \\
		\bottomrule
	\end{tabular}
	\caption{Common Terraform built-in function categories}
\end{table}

\begin{definition}{Block}{}
	\begin{alist}
		\item containers for other content
		\item usually represent the configuration of some kind of object, like a
		resource.
		\item have a block type, zero or more labels, and body
		that contains any number of arguments and nested blocks.
		\item most of terraform's features are controlled by top-level blocks in
		a configuration file.
	\end{alist}

	block types
	\begin{multicols}{4}
		\begin{enumerate}[label = {(\arabic*)}]
			\item data
			\item import
			\item locals
			\item module
			\item output
			\item provider
			\item provisioner
			\item resource
			\item terraform
			\item variable
		\end{enumerate}
	\end{multicols}
\end{definition}

config blocks

\begin{definition}{Resource Block}{}
	\begin{verbatim}
resource "aws_instance" "helloworld" {...}
element  type           name         {...}
          |_________________________|
                     \/
                 identifier
    \end{verbatim}
\end{definition}

\begin{definition}{Data Block}{}
	Like a read-only version of Resource blocks. Used to retrieve data about
	resources that already exists and are managed by an external entity.
	\begin{enumerate}[label = {(\arabic*)}]
		\item nested blocks
		\item attributes
	\end{enumerate}
\end{definition}

\begin{definition}{Argument}{}
	\begin{alist}
		\item assign a value to a name.
		\item appear within blocks.
	\end{alist}
\end{definition}

\begin{definition}{expression}{}
	\begin{alist}
		\item represent a value, either literally or by referencing and combining other values.
		\item appear as values for arguments, or within other expressions.
	\end{alist}
\end{definition}

\begin{definition}{Module}{}
	self-contained packages of code that allow you to create reusable components
	by grouping related resources together. You don’t have to know how a module
	works to be able to use it; you just have to know how to set inputs and
	outputs.
\end{definition}

\begin{definition}{Module}{}
	Think of a module as being sort of like a function with side effects, meaning:
	\begin{enumerate}[label = {(\arabic*)}]
		\item function has inputs Terraform calls these input variables.
		\item function has outputs Terraform calls these output values.
		\item function does things other than producing output values For example, the function might provision a resource.
		\item invoke another terraform module like a function call
	\end{enumerate}
\end{definition}


\end{document}
